%%%%%%%%%%%%%%%%%%%%%%%%%%%%%%%%%%%%%%%%%%%%%%%%%%%%%%%%%%%%%%%%%%%%%%%%%%%%%%%%

\pagestyle{empty}
\begin{abstract}
    
  El trabajo consiste en el diseño, implementación de una aplicación para ayudar en la gestión de
  procesos de negocio de una empresa viticultora, en específico en el proceso de la recogida de la uva
  (vendimia).

  Durante el proceso de vendimia la plantilla de empleados crece enormemente por lo que una
  aplicación para la ayuda de la gestión de los trabajos de los empleados puede ayudar mucho en el
  control y trazabilidad de la uva y del trabajo realizado.

  La aplicación tendrá un catálogo de zonas de recogida y líneas de parras asociadas a cada zona, y
  estas líneas de parras con datos sobre el tipo de uva, el tipo de formación empleados en la línea,
  edad, metros de línea, etc. Esta información podrá ser creada y modificada por empleados de la
  empresa con rol de administrador.Para la gestión del trabajo de la vendimia, los capataces podrán, mediante el uso de la aplicación,
  informar del trabajo realizado en las distintas líneas de parras, tanto de tareas de mantenimiento
  previa a la vendimia, como de la recogida de la uva.

  Los tractoristas tendrán un rol especial, ya que serán notificados una vez se trabaje por completo una
  cantidad determinada de líneas de parras, con información detallada de la zona y líneas que deben
  recoger.

  Para la facilidad de uso de la aplicación por parte de los capataces, estos podrán registrar el trabajo
  de cada línea de parras con un código QR, que habrá en cada línea de parras.

  Los capataces podrán utilizar la aplicación para añadir trabajos asociados a cada una de las líneas de
parras, asignando recursos (personal de vendimia), tipo de trabajo, y opcionalmente añadir
comentarios sobre el trabajo realizado en una línea de parras. Para facilitar el uso se puede leer un
código QR para iniciar este proceso, aunque también tendrá la posibilidad de buscar la zona y línea
manualmente.

Habrá tres tipos de roles; administrador, capataz y tractoristas. Un mismo usuario podrá ejercer rol
de capataz y tractorista. Se requerirá autenticación de estos usuarios.
La aplicación constará de un backend y un frontend. El backend estará implementado en Java, y será
accesible mediante una API REST basada en Spring y desarrollada con una metodología API First con
el uso de la herramienta OpenAPI.



\vspace*{25pt}
\begin{segundoresumo}
  
  The work consists in the design and implementation of an application to help in the management of
  business processes of a wine-growing company, specifically in the process of grape harvesting (grape harvest).

  During the harvesting process, the number of employees grows enormously, so an application for the
  application to help manage the work of the employees can be of great help in the control and traceability of the grapes.
  control and traceability of the grapes and the work carried out.

  The application shall have a catalogue of harvesting zones and vine lines associated to each zone, and
  these vine lines with data on the type of grape, the type of formation used on the line, age, metres of line, etc. 
  This information can be created and modified by company employees with the role of administrator.
  For the management of harvest work, foremen will be able, by using the application, to report on the work carried out on the different lines,
  report on the work carried out on the different lines of vines, both in terms of maintenance tasks prior to the harvest, as well as
  tasks prior to the harvest, as well as grape harvesting.

  Tractor drivers will have a special role, as they will be notified once a certain number of vine lines have been completely worked.
  number of lines of vines are completely worked, with detailed information on the area and lines where to pick up the boxes loaded with grapes.

  For the ease of use of the application by the foremen, they will be able to register the work of each line of vines with a QR code, 
  which will be on each line of vines.

  Foremen will be able to use the application to add jobs associated with each of the lines of vines, 
  assigning resources (harvesting personnel, type of work, and optionally adding
  resources (harvesting personnel), type of work, and optionally add comments on the work carried out on a line of vines.
  comments on the work carried out on a line of vines. For ease of use, a
  QR code can be read to start this process, although you can also search for the area and line manually.
  
  There will be three types of roles; administrator, foreman and tractor drivers. The same user will be able to play the role of
  foreman and tractor driver. Authentication of these users will be required.
  The application will consist of a backend and a frontend. The backend will be implemented in Java, and it will be
  accessible via a REST API based on Spring and developed with an API First methodology using the OpenAPI tool.
\end{segundoresumo}
\vspace*{25pt}
\begin{multicols}{2}
  \begin{description}
  \item [\palabraschaveprincipal:] \mbox{} \\[-20pt]
    \begin{itemize}
      \item JPA
      \item Hibernate
      \item Spring
      \item Flutter
      \item OpenAPI
      \item OAS
      \item OpenAPI Generator
      \item Scrumban
      \item Funcionalidad
      \item Vendimia
      \item Administrador
      \item Capataz
      \item Tractorista 
    \end{itemize}
  \end{description}
  \begin{description}
  \item [\palabraschavesecundaria:] \mbox{} \\[-20pt]
    \begin{itemize}
      \item JPA
      \item Hibernate
      \item Spring
      \item Flutter
      \item OpenAPI
      \item OAS
      \item OpenAPI Generator
      \item Scrumban
      \item Feature
      \item Harvest
      \item Administrator
      \item Foreman
      \item Tractor driver
    \end{itemize}

  \end{description}
  \end{multicols}
  
\end{abstract}
\pagestyle{fancy}

%%%%%%%%%%%%%%%%%%%%%%%%%%%%%%%%%%%%%%%%%%%%%%%%%%%%%%%%%%%%%%%%%%%%%%%%%%%%%%%%
