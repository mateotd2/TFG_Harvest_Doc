\chapter{Metodologias}
\label{chap:metodologias}


\lettrine{E}{ntendemos} como metodología como un conjunto ordenado, en cuanto tiene una pauta temporal y determina importancia y secuencialidad y bien estructurado, que determina planteamientos generales para desarrollar el trabajo. En segundo lugar, filosofías o políticas que pueden abordar el modo en el que pretendemos adecuar tecnología con la definición de las soluciones y que puede dar lugar al desarrollo de las herramientas y técnicas que pueden confeccionar una solución. 

\section{Metodologías agiles}

La metodología agiles se basan en el manifiesto ágil, creado en 2001 por un grupo de desarrolladores de software que buscaban una forma más efectiva de desarrollar el software, el cual tiene como principios:

\begin{enumerate}
    \item 1.	Individuos e interacciones sobre procesos y herramientas: En la que la colaboración entre las personas es más importante que seguir estrictamente los procesos o depender de herramientas.
    \item 2.	Software funcionando sobre documentación extensiva: el objetivo principal es entregar software funcional.
    \item 3.	Colaboración con el cliente sobre negociación contractual: Comunicación constante con el cliente es esencial para asegurar el producto final.
    \item 4.	Respuesta ante el cambio sobre seguir un plan: La capacidad de adaptarse y responde a los cambios.
\end{enumerate}

Normalmente las metodologías agiles se basan en un desarrollo incremental, lo cual se traduce en un software entregable que puede ser explotado. Este tipo de desarrollos nos permiten entregar pocas funcionalidades, pero evidentes y fácilmente adaptables.

Las metodologías agiles permiten a los equipos de desarrollo adaptarse rápidamente a los cambios en los requisitos y prioridades del cliente, mejorar la colaboración y comunicación y entregar productos de calidad de manera continua.

Se explicará en detalle las metodologías más importantes, Scrum y Kanban.
